\documentclass{beamer}

%\usepackage[table]{xcolor}
\mode<presentation> {
  \usetheme{Boadilla}
%  \usetheme{Pittsburgh}
%\usefonttheme[2]{sans}
\renewcommand{\familydefault}{cmss}
%\usepackage{lmodern}
%\usepackage[T1]{fontenc}
%\usepackage{palatino}
%\usepackage{cmbright}
  \setbeamercovered{transparent}
\useinnertheme{rectangles}
}
%\usepackage{normalem}{ulem}
%\usepackage{colortbl, textcomp}
\setbeamercolor{normal text}{fg=black}
\setbeamercolor{structure}{fg= black}
\definecolor{trial}{cmyk}{1,0,0, 0}
\definecolor{trial2}{cmyk}{0.00,0,1, 0}
\definecolor{darkgreen}{rgb}{0,.4, 0.1}
\usepackage{array}
\beamertemplatesolidbackgroundcolor{white}  \setbeamercolor{alerted
text}{fg=red}

\setbeamertemplate{caption}[numbered]\newcounter{mylastframe}

%\usepackage{color}
\usepackage{tikz}
\usetikzlibrary{arrows}
\usepackage{colortbl}
%\usepackage[usenames, dvipsnames]{color}
%\setbeamertemplate{caption}[numbered]\newcounter{mylastframe}c
%\newcolumntype{Y}{\columncolor[cmyk]{0, 0, 1, 0}\raggedright}
%\newcolumntype{C}{\columncolor[cmyk]{1, 0, 0, 0}\raggedright}
%\newcolumntype{G}{\columncolor[rgb]{0, 1, 0}\raggedright}
%\newcolumntype{R}{\columncolor[rgb]{1, 0, 0}\raggedright}

%\begin{beamerboxesrounded}[upper=uppercol,lower=lowercol,shadow=true]{Block}
%$A = B$.
%\end{beamerboxesrounded}}
\renewcommand{\familydefault}{cmss}
%\usepackage[all]{xy}

\usepackage{tikz}
\usepackage{lipsum}

 \newenvironment{changemargin}[3]{%
 \begin{list}{}{%
 \setlength{\topsep}{0pt}%
 \setlength{\leftmargin}{#1}%
 \setlength{\rightmargin}{#2}%
 \setlength{\topmargin}{#3}%
 \setlength{\listparindent}{\parindent}%
 \setlength{\itemindent}{\parindent}%
 \setlength{\parsep}{\parskip}%
 }%
\item[]}{\end{list}}
\usetikzlibrary{arrows}
%\usepackage{palatino}
%\usepackage{eulervm}
\usecolortheme{lily}

\newtheorem{com}{Comment}
\newtheorem{lem} {Lemma}
\newtheorem{prop}{Proposition}
\newtheorem{thm}{Theorem}
\newtheorem{defn}{Definition}
\newtheorem{cor}{Corollary}
\newtheorem{obs}{Observation}
 \numberwithin{equation}{section}


\title[Methodology I] % (optional, nur bei langen Titeln nötig)
{Math Camp}

\author{Justin Grimmer}
\institute[University of Chicago]{Associate Professor\\Department of Political Science \\  University of Chicago}
\vspace{0.3in}

\date{August 28th, 2017}

\begin{document}
\begin{frame}
\titlepage
\end{frame}


\begin{frame}

\Large
$<$ Course $>$

\end{frame}


\begin{frame}
\frametitle{The Systematic Analysis of Politics}

\alert{Social Science}: systematic analysis of society\pause \invisible<1>{ (Political Science: who gets what, when, and how).  }\\
\pause
\invisible<1-2>{\alert{Methodology}: Develop and disseminate tools to make inferences about society}\pause
\begin{itemize}
\item[-] \invisible<1-3>{Mathematical models of social world} \pause
\item[-] \invisible<1-4>{Probability and Statistics used across sciences} \pause
\end{itemize}

\invisible<1-5>{This class (introduction):} \pause
\begin{itemize}
\item[-] \invisible<1-6>{Math Camp: Develop Tools for Analysis} \pause
\item[-] \invisible<1-7>{Probability theory: systematic model of randomness}
%\item[-] \invisible<1-8>{Statistical inference: learning from data}
\end{itemize}

\end{frame}


\begin{frame}
\frametitle{Course Goals}

First stop in methodology sequence\\
\invisible<1>{\alert{Big Goal}: prepare you to make \alert{discoveries} about social world} \pause \\
\invisible<1-2>{\alert{Proximate Goals}} \pause
\begin{itemize}
\invisible<1-3>{\item[1)] Mathematical tools to comprehend and use statistical methods} \pause
\invisible<1-3>{\item[2)] Foundation in probability theory/analytic reasoning} \pause
%\invisible<1-4>{\item[3)] Deep understanding of basic concepts in statistical inference } \pause
\invisible<1-4>{\item[3)] Practical Computing Tools: {\tt R} }
\end{itemize}


\end{frame}

\begin{frame}
\frametitle{Course Staff}

Me: Justin Grimmer \pause
\begin{itemize}
\invisible<1>{\item[-] Office: Pick Hall 423 } \pause
\invisible<1-2>{\item[-] Email: grimmer@uchicago.edu} \pause
\invisible<1-3>{\item[-] Cell: 617-710-6803} \pause
\invisible<1-4>{\item[-] Office Hours: I'm generally here all the time (9am to 5pm), just stop by [but if you need to see me with 100\% probability, schedule a visit]} \pause
\end{itemize}


\end{frame}


\begin{frame}
\frametitle{TA Info}


\begin{itemize}
     \item  Joshua Mausolf \\
              Ryan Hughes \\
     \item We will hold twice weekly labs, that will occur in this room from 130-300pm (or so)  \\
     \item Github for class:
\url{github/justingrimmer/MathCamp17} \\

\end{itemize}

\end{frame}



\begin{frame}
\frametitle{Prerequisites}

\alert{No Formal Prerequisites} \pause \\
\invisible<1>{\large BUT } \pause \\
\normalsize
\begin{itemize}
\invisible<1-2>{\item[-] Successful students will know differential and integral calculus} \pause
\begin{itemize}
\invisible<1-3>{\item[1)] Limits (intuitive)} \pause
\invisible<1-4>{\item[2)] Derivatives (tangent lines, differentiation rules)} \pause
\invisible<1-5>{\item[3)] Integrals (fundamental theorem of calculus/antidifferentiation rules} \pause
\end{itemize}
\invisible<1-6>{\item[-] \alert{We are here to help}} \pause
\begin{itemize}
\invisible<1-7>{\item[-] No mystery to learning math: just hard work} \pause
\invisible<1-8>{\item[-] Political science increasingly requires math} \pause
\invisible<1-9>{\item[-] Empirical: calculus and linear algebra} \pause
\invisible<1-10>{\item[-] Quantitative Methodologist: Real Analysis and
  Grad level statistics} \pause
\invisible<1-11>{\item[-] Formal Theory: Real Analysis (through measure theory),
  Topology}
\end{itemize}
\end{itemize}
\end{frame}


\begin{frame}
\frametitle{Evaluation}
You're not taking this class for a grade\pause \invisible<1>{ $\leadsto$ that shouldn't matter: }\pause
\begin{itemize}
\invisible<1-2>{\item[-]  Math Camp Exam}
%\invisible<1-2>{\item[-] 40\%: Homework assignments} \pause
%\invisible<1-3>{\item[-] 15\%: Midterm} \pause
%\invisible<1-4>{\item[-] 25\% Final Exam} \pause
%\invisible<1-5>{\item[-] 10\%: Participation} \pause
\end{itemize}

\invisible<1-3>{\alert{Grad School Irony}} \pause \invisible<1-4>{ Or: \alert{How I Learned to Stop Worrying and Love C's} } \pause
\begin{itemize}
\invisible<1-5>{\item[-] Grades no longer matter} \pause
\invisible<1-6>{\item[-] Learn as much material as possible} \pause
\invisible<1-7>{\item[-] \alert{If you truly only care about learning material, you'll get
    amazing grades}} \pause
\end{itemize}

\invisible<1-8>{\Large \alert{No incompletes in this class}}


\end{frame}

%\begin{frame}
%\frametitle{Math Camp Exam}

%September 24th. 11 am.  Closed book, closed notes\\

%Excuses: only acts of God/official Stanford excuses.

%\end{frame}


\begin{frame}
\frametitle{Homework}
Math camp: assigned daily $\leadsto$ Mechanics of solving problems\\
Lab Assignment: Twice weekly assignments, help you develop computational and mathematical skills.


%During quarter: assigned each Friday (at Section) \pause \\
%\invisible<1>{To be turned in each Friday (next Section)} \pause \\
%\begin{itemize}
%\invisible<1-2>{\item[-] Computing: \alert{R} and \LaTeX (to be discussed below)} \pause
%\invisible<1-3>{\item[-] Collaboration: encouraged, but write up your own code +
%  answers} \pause
%\end{itemize}
%\invisible<1-4>{\large Challenging: this class is an investment--work hard, will pay
%  dividends}
%

\end{frame}

\begin{frame}
\frametitle{Computing/Homeworks}

Greatest scientific discovery of 20th Century: \pause \\
\invisible<1>{Powerful personal computer (standardize science)} \pause \\
\invisible<1-2>{1956: \$10,000 megabyte}\\ \pause \invisible<1-3>{2015:
  $<<<$ \$ 0.0001 per megabyte } \pause \\
\invisible<1-4>{Statistical Computing: \alert{R}} \pause
\begin{itemize}
\invisible<1-5>{\item[-] {\tt R}: Scripting language} \pause
\invisible<1-6>{\item[-] Flexible}\pause\invisible<1-7>{, Cutting Edge Software}\pause\invisible<1-8>{, great visualization tools}\pause\invisible<1-9>{
  and makes learning other programs easier}\pause
\item[-] \invisible<1-10>{More start up costs than STATA, but more
    payoff}\pause
\end{itemize}
\invisible<1-11>{Paper writeup: \LaTeX }\pause
\begin{itemize}
\invisible<1-12>{\item[-] Hard to write equations in Word: } \pause
\invisible<1-13>{\item[-] Relatively easy in \LaTeX \\} \pause
\invisible<1-14>{$f(x) = \frac{\exp(-\frac{(x - \mu)^2}{2\sigma^2} )}{ \sqrt{2\pi \sigma^2}} $}\pause
\invisible<1-15>{\item[-] Tables/Figures/General type/Nice Presentations setting: easier in \LaTeX} \pause
\invisible<1-16>{\item[-] \alert{If you use start using \LaTeX, you'll soon love it} }
\end{itemize}
\end{frame}

%
%\begin{frame}
%\frametitle{Midterm}
%
%Closed book, in class exam.
%\large
%October 22.
%
%\normalsize
%Excuses: act of God/official Stanford excuses only
%
%
%
%\end{frame}
%
%
%
%\begin{frame}
%\frametitle{Final Exam}
%
%Open book, take home exam.  (no collaboration)
%
%\end{frame}

%\begin{frame}
%\frametitle{Participation}
%
%Participation in class/section is \alert{essential}\\
%Questions most important component of participation
%\begin{itemize}
%\item[-] In Class (slow down pace, which is good)
%\item[-] Class Listserve (www.piazza.com) ask/answer questions
%\item[-] Office Hours
%\end{itemize}
%
%\end{frame}


\begin{frame}
\frametitle{Course Books}


\begin{itemize}
\item[1)] Simon, Carl and Blume, Lawrence (SB).  Mathematics for Economists.
\item[2)] Bertsekas, Dimitri P. and Tsitsiklis, John (BT) Introduction to Probability Theory (second edition)
%\item[2)] Probability Theory (Mathematical model of uncertainty), Ross
%\begin{itemize}
%\item[-] Mathematically sophisticated (assumes solid calculus
%  background)
%\item[-] Read carefully, talk to us with questions [before class!]
%\item[-] Examples: work through as many as you can
%\end{itemize}
%\item[3)] Probability and Statistics Degroot and Schervish
%\begin{itemize}
%\item[-] Inference: how to learn about stuff from data
%\end{itemize}
%\item[4)] Visualizing Data, Cleveland
%\begin{itemize}
%\item[-] Inference: not just from statistical models
%\item[-] Visualization: not just from presentation, vital for learning
%\item[-] Introduction to what \alert{works} and doesn't in visualization
%\end{itemize}
%\item[5)] R programming: Teetor
%\begin{itemize}
%\item[-] Help you learn to program
%\item[-] Teach you how to fish (learn new code yourself)
%\end{itemize}
\end{itemize}

\end{frame}





\begin{frame}
\frametitle{Life in Graduate School/Academy}

Three part mixture: \pause


\begin{tikzpicture}


\invisible<1>{\node (george) at (-6,4) [] {\scalebox{0.5}{\includegraphics{Strait.jpg} }} ;
\node (gtext) at (-6, 1) [] {George Strait} ; }\pause


\invisible<1-2>{\node (kayne) at (-3, 3.5) [] {\scalebox{0.5}{\includegraphics{kanye.jpg}} } ;
\node (kayne) at (-3, 0.5) [] {Kanye West}; } \pause


\invisible<1-3>{\node (steve) at (-1, 3) [] {\scalebox{1.25}{\includegraphics{Steve.jpg}}} ;
\node (stevet) at (-1, 0) [] {\alert{Steve Prefontaine}}; }

\end{tikzpicture}


\end{frame}


\begin{frame}
\frametitle{$\frac{1}{3}$ George Strait}

\begin{columns}[]
\column{0.5\textwidth}

\scalebox{0.75}{\includegraphics{Strait.jpg}}


\column{0.5\textwidth}
\pause
\begin{itemize}
\invisible<1>{\item[-] Amarillo By Morning [Terry Stafford 1973, George Strait 1982]} \pause
\invisible<1-2>{\item[-] Ostensibly: song about rodeo cowboys} \pause
\invisible<1-3>{\item[-] Really: song about being academic } \pause
\invisible<1-4>{\item[-] ``I ain't got a dime/but what I got is mine/I ain't rich/ but lord I'm free"} \pause
\invisible<1-5>{\item[-] Academics ain't rich (counterfactually)} \pause
\invisible<1-6>{\item[-] But (lord) we're free}\pause
\invisible<1-7>{\item[-] If you're good at methods, you'll be more
  rich [in expectation] and equally free}
\end{itemize}



\end{columns}




\end{frame}

\begin{frame}
\frametitle{$\frac{1}{3}$ Kayne West}


\begin{columns}[]

\column{0.5\textwidth}
\scalebox{0.5}{\includegraphics{kanye.jpg}}


\column{0.5\textwidth}
\pause
\begin{itemize}
\invisible<1>{\item[-] Deal with explicit criticism (part of Hip/Hop
  culture) } \pause
\invisible<1-2>{\item[-] On masterpiece album \alert{My Beautiful Dark Twisted Fantasy}} \pause
\invisible<1-3>{\item[-] ``Screams from the haters, got a nice ring to it/I guess
  every superhero needs his theme music'' } \pause
\invisible<1-4>{\item[-] Kid Cudi: ``These motherf**kers can't fathom the wizadry'' } \pause
\invisible<1-5>{\item[-] Academics: intense criticism of ideas } \pause
\invisible<1-6>{\item[-] \alert{Very rarely will you be told you're doing a great job}} \pause
\invisible<1-7>{\item[-] Self confidence: believe in work}
\end{itemize}

\end{columns}
\end{frame}


\begin{frame}
\frametitle{$\frac{1}{3}$ Steve Prefontaine}

``It's not a sprint, it's a marathon".


\only<1>{\scalebox{2}{\includegraphics{Steve.jpg}}}
\pause
\only<2->{
\begin{itemize}
\invisible<1>{\item[-] World class distance running: it is
  \alert{hard}} \pause
\invisible<1-2>{\item[-] But not for the obvious reasons } \pause
\invisible<1-3>{\item[-] Marathon: 4:40 minute mile, for 26.2 miles.} \pause
\invisible<1-4>{\item[-] How to train?} \pause
\begin{itemize}
\invisible<1-5>{\item[-] Old way: get in shape (run far) rely on adrenaline in race } \pause
\invisible<1-6>{\item[-] Now: races more tactical \alert{and agonizing} } \pause
\invisible<1-7>{\item[-] Need to prepare for agony } \pause
\end{itemize}
\invisible<1-8>{\item[-] Mantra: \alert{sustained agony}} \pause
\invisible<1-9>{\item[-] Graduate School/Academics: \alert{Sustained Agony}} \pause
 \end{itemize}
}


\end{frame}

\begin{frame}
\frametitle{$\frac{1}{3}$ Steve Prefontaine}

\pause
\invisible<1>{{\huge Not crazy to work \alert{40} hours on methods \alert{alone}}}

\pause

\begin{itemize}
\invisible<1-2>{\item[-] Methods $\leadsto$ skills use for rest of career} \pause
\invisible<1-3>{\item[-] Methods $\leadsto$ often takes deep thinking, practice} \pause
\end{itemize}

\invisible<1-4>{{\huge \alert{TAKE BREAKS!}}} \pause

\begin{itemize}
\invisible<1-5>{\item[-] Regular physical activity $\leadsto$ improve focus} \pause
\invisible<1-6>{\item[-] Time away from lab $\leadsto$ more productive when back}
\end{itemize}


\end{frame}




\begin{frame}
\frametitle{$\frac{1}{3}$ Steve Prefontaine}

Why work so hard?\pause
\begin{itemize}
\invisible<1>{\item[-] \alert{You are all smart}} \pause
\invisible<1-2>{Really Smart} \pause \invisible<1-3>{Mother-in-law
  brags about you smart} \pause
\invisible<1-4>{\item[-] Everyone entering graduate school at top programs this fall} \pause
\invisible<1-5>{\item[-] Success: \alert{work}}\pause
\invisible<1-6>{\item[-] Treat grad school like a job}\pause
\invisible<1-7>{\item[-]Who gets ahead? who gets the most work done on the smartest ideas}
\end{itemize}

\end{frame}


\begin{frame}
\frametitle{Preliminaries}

What can you learn in a math camp?\pause
\begin{itemize}
\invisible<1>{\item[1)] Introduction to more sophisticated mathematics (notation)}\pause
\invisible<1-2>{\item[2)] Getting acquainted with proof techniques and proofs} \pause
\invisible<1-3>{\item[3)] I'm going to introduce ideas/example problems common in research that will help with your seminar} \pause
\invisible<1-4>{\item[4)] \alert{This will not substitute for a richer math background} and \alert{we won't expect it to}} \pause
\end{itemize}
\large
\invisible<1-5>{\alert{Do not let yourself get lost}.} \pause  \\
\invisible<1-6>{If \alert{at.}}\pause \invisible<1-7>{ \alert{any.} }\pause\invisible<1-8>{\alert{point.} } \pause \invisible<1-9>{you have a question} \pause
\Large
\invisible<1-10>{\alert{please ask !}} \pause \\
\invisible<1-11>{Smartest people ask the most questions!}

\end{frame}





\begin{frame}
\Huge
Let's get to work
\end{frame}



\begin{frame}
\frametitle{Sets}

A \alert{set} is a collection of objects.

\begin{eqnarray}
A & = & \{1, 2, 3\} \nonumber  \\
B  & = & \{4, 5, 6\}\nonumber  \\
C  & = & \{ \text{First year cohort} \}\nonumber  \\
D & = & \{ \text{U of Chicago Faculty} \}\nonumber
\end{eqnarray}





\end{frame}

\begin{frame}



\begin{defn} If $A$ is a set, we say that $x$ is an element of $A$ by writing, $x \in A$.  If $x$ is not an element of $A$ then, we write $x \notin A$.
\end{defn}

\pause
\begin{itemize}
\invisible<1>{\item[-] $ 1 \in \{ 1, 2, 3\}$}\pause
\invisible<1-2>{\item[-] $4 \in \{4, 5, 6\} $}\pause
\invisible<1-3>{\item[-] $\text{Will} \notin \{ \text{First year cohort} \} $}\pause
\invisible<1-4>{\item[-] $\text{Justin} \in \{ \text{U Chicago Faculty} \}$}\pause
\end{itemize}

\invisible<1-5>{Why Care?} \pause
\begin{itemize}
\invisible<1-6>{\item[-] Sets are necessary for probability theory} \pause
\invisible<1-7>{\item[-] Defining \alert{set} is equivalent ot choosing population of interest (usually)}
\end{itemize}



\end{frame}



\begin{frame}


\begin{defn} If $A$ and $B$ are sets, then we say that $A = B$ if, for all $x \in A$ then $x \in B$ and for all $y \in B$ then $y \in A$.
\end{defn}

\pause
\begin{itemize}
\invisible<1>{\item[-] Test to determine equality:} \pause
\begin{itemize}
\invisible<1-2>{\item[-] Take all elements of $A$, see if in $B$} \pause
\invisible<1-3>{\item[-] Take all elements of $B$, see if in $A$} \pause
\end{itemize}
\end{itemize}

\invisible<1-4>{\begin{defn} If $A$ and $B$ are sets, then we say that $A \subset B$ is, for all $x \in A$, then $x \in B$.
\end{defn}} \pause

\invisible<1-5>{\alert{Difference between definitions? }}


\end{frame}


\begin{frame}
\begin{thm} Let $A$ and $B$ be sets.  If $A$ = $B$ then $A \subset B$ and $B \subset A$ \end{thm}
\pause
\invisible<1>{ \begin{proof}  Suppose $A = B$.  By definition, if $x \in A$ then $x \in B$.  So $A \subset B$.  Again, by definition, if $y \in B$ then $y \in A$.  So $B \subset A$. \end{proof}} \pause
\invisible<1-2>{
\begin{thm}
Let $A$ and $B$ be sets. If $A \subset B$ and $B \subset A$ then $A = B$ \end{thm} } \pause
\invisible<1-3>{ \begin{proof} Suppose $A \subset B$  and that $B \subset A$.  For all $x \in A$, then $x \in B$.  And for all $y \in B$, $y \in A$.  Or, every element in $A$ is in $B$ and each element of $B$ is in $A$.  $A = B$. \end{proof} }



\end{frame}



\begin{frame}


\begin{thm} Let $A$ and $B$ be sets.  Then $A$ = $B$ \alert{if and only if} $A \subset B$ and $B \subset A$. \end{thm}
\pause
\invisible<1>{\begin{proof} $\Rightarrow$ Suppose $A = B$.  By definition, if $x \in A$, $x \in B$.  So $A \subset B$.  Again, by definition, if $y \in B$ then $y \in A$.  So $B \subset A$.  \\} \pause
\invisible<1-2>{$\Leftarrow$ Suppose $A \subset B$  and that $B \subset A$.  For all $x \in A$, then $x \in B$.  And for all $y \in B$, $y \in A$.  Or, every element in $A$ is in $B$ and each element of $B$ is in $A$.  $A = B$.  } \pause
\end{proof}

\invisible<1-3>{When a proof says \text{if and only if} it is showing two things.  } \pause
\begin{itemize}
\invisible<1-4>{\item[-] \alert{If} or that a condition is \alert{sufficient} } \pause
\invisible<1-5>{\item[-] \alert{Only If} or that a condition is necessary } \pause
\end{itemize}

\invisible<1-6>{Example of sufficient, but not necessary} \pause
\begin{itemize}
\invisible<1-7>{\item[-] If candidate wins the electoral college, then president (can be president through vote of House too)} \pause
\end{itemize}
\invisible<1-8>{Example of necessary, but not sufficient} \pause
\begin{itemize}
\invisible<1-9>{\item[-] Only if a candidate is older than 35 can s/he be president (but clearly not sufficient)} \pause
\end{itemize}


\invisible<1-10>{\alert{Proofs} : we're going to work hard on proofs. The only way to get better is to practice.  }

\end{frame}


\begin{frame}
\frametitle{Contradiction}

\pause
\begin{itemize}
\invisible<1>{\item[-] Many ways to prove the same theorem.  } \pause
\invisible<1-2>{\item[-] \alert{Contradiction}: assume theorem is false, show that this leads to logical contradiction} \pause
\invisible<1-3>{\item[-] \alert{Indirect proof}: setting up proof hardest part } \pause
\end{itemize}

\invisible<1-4>{\begin{thm} Let $A$ and $B$ be sets.  Then $A$ = $B$ \alert{if and only if} $A \subset B$ and $B \subset A$. \end{thm} } \pause
\invisible<1-5>{\begin{proof} $\Rightarrow$ Suppose $A = B$.  By definition, if $x \in A$, $x \in B$.  So $A \subset B$.  Again, by definition, if $y \in B$ then $y \in A$.  So $B \subset A$. } \pause  \\
\invisible<1-6>{$\Leftarrow$ Suppose $A \subset B$  and that $B \subset A$. Now, by way of contradiction, suppose that $A \ne B$.  $A \ne B$ only if there is $x \in A $ and $x \notin B$ or if $y \in B$ and $y \notin A$.  But then, either $A \not\subset B$ or $B \not\subset A$, contradicting our initial assumption.  }
\end{proof}


\end{frame}


\begin{frame}
\frametitle{Set Builder Notation}

\begin{itemize}
\item[-] Some famous sets
\begin{itemize}
\item[-] $J$ = $\{1, 2, 3, \hdots \}$
\item[-] $Z$ = $\{\hdots, -2, -1, 0, 1, 2, \hdots, \}$
\item[-] $\Re$ = \alert{real numbers} (more to come about this)
\end{itemize}
\item[-] Use \alert{set builder notation} to identify subsets
\begin{itemize}
\item[-] $[a, b]  = \{x: x \in \Re \text{ and } a \leq x \leq b \}$
\item[-] $(a, b]  = \{x: x \in \Re \text{ and } a < x \leq b \}$
\item[-] $[a, b)  = \{x: x \in \Re \text{ and } a \leq x <  b \}$
\item[-] $(a, b)  = \{x: x \in \Re \text{ and } a < x < b \}$
\item[-] $\emptyset$
\end{itemize}
\end{itemize}



\end{frame}


\begin{frame}
\frametitle{Set Operations}

We can build new sets with \alert{set operations}.  \pause


\invisible<1>{\begin{defn}  Suppose $A$ and $B$ are sets.  Define the \alert{Union} of sets $A$ and $B$ as the new set that contains all elements in set $A$ \alert{or} in set $B$.  In notation,
\begin{eqnarray}
C & = & A \cup B \nonumber \\
   & = &\{x: x \in A \text{ or } x \in B \} \nonumber
 \end{eqnarray}
\end{defn}} \pause

\begin{itemize}
\invisible<1-2>{\item[-] $A = \{1, 2, 3\}, B = \{3, 4, 5\}$, then $C = A \cup B  = \{ 1, 2, 3, 4, 5\}$} \pause
\invisible<1-3>{\item[-] $D  = \{\text{First Year Cohort} \}, E = \{\text{Me} \}$, then $F = D \cup E = \{ \text{First Year Cohort, ME} \} $}
\end{itemize}

\end{frame}


\begin{frame}
\frametitle{Set Operations}

\begin{defn} Suppose $A$ and $B$ are sets.  Define the \alert{Intersection} of sets $A$ and $B$ as the new that contains all elements in set $A$ \alert{and} set $B$.  In notation,
\begin{eqnarray}
C  & = & A \cap B \nonumber \\
  & = & \{x: x \in A \text{ and } x \in B \} \nonumber
  \end{eqnarray}
\end{defn}
\pause
\begin{itemize}
\invisible<1>{\item[-] $A =\{1, 2, 3\}, B = \{3, 4, 5\}$, then, $C = A \cap B = \{3\} $ } \pause
\invisible<1-2>{\item[-] $D = \{\text{First Year Cohort} \}, E = \{\text{Me} \}$, then $F = D \cap E = \emptyset $}
\end{itemize}


\end{frame}




\begin{frame}
\frametitle{Some Facts about Sets (No Venn Diagrams!!!)}

\begin{itemize}
\item[1)] $A \cap B = B \cap A$
\invisible<1-4>{\item[2)] $A \cup B = B \cup A$}
\invisible<1-4>{\item[3)] $(A \cap B) \cap C = A \cap (B \cap C)$}
\invisible<1-4>{\item[4)] $(A \cup B) \cup C = A \cup (B \cup C) $}
\invisible<1-2>{\item[5)] $A \cap (B \cup C) = (A \cap B) \cup (A \cap C)$}
\invisible<1-4>{\item[6)] $A \cup (B \cap C) = (A \cup B) \cap (A \cup C) $}
\end{itemize}


\only<2>{\begin{proof}
This fact (theorem) says that the \alert{set} $A \cap B$ is equal to the set $B \cap A$.  We can use the definition of equal sets to test this.  Suppose $x \in A \cap B$.  Then $x \in A$ and $x \in B$.  By definition, then, $x \in B \cap A$.  Now, suppose $y \in B \cap A$. Then $y \in B$ and $y \in A$.  So, by definition of intersection $y \in A \cap B$.  This implies $ A \cap B = B \cap A$ \end{proof}}

\only<4>{\begin{proof}
Suppose $x \in A \cap (B \cup C)$.  Then $x\in B$ or $x \in C$ \alert{and} $x \in A$.  This implies that $x \in (A \cap B)$ or $ x \in (A \cap C)$.  Or, $ x \in (A \cap B) \cup (A \cap C)$.  Now, suppose $y \in (A \cap B) \cup (A \cap C) $.  Then, $ y \in A$ and $y \in B$ or $y \in C$.  Well, this implies $y \in A \cap (B \cup C)$.  And we have established equality
\end{proof} }

\only<5>{Break into groups, derive for the remaining facts }



\pause \pause \pause \pause


\end{frame}


\begin{frame}
\frametitle{Ordered Pair}
You've seen an \alert{ordered pair} before,
\begin{equation}
(a, b)  \nonumber
\end{equation}


\begin{defn}
Suppose we have two sets, $A$ and $B$.  Define the \alert{Cartesian product} of $A$ and $B$, $A \times B$ as the set of all ordered pairs $(a, b)$, where $a \in A$ and $b \in B$.  In other words,
\begin{eqnarray}
A \times B & = & \{(a, b): a \in A \text{ and } b \in B \} \nonumber
\end{eqnarray}
\end{defn}


Example: \\
$A = \{1, 2\}$ and $B = \{3, 4\}$, then, \\
$A \times B = \{ (1, 3); (1, 4); (2, 3); (2, 4) \} $

\end{frame}


\begin{frame}
\frametitle{Function}
Start with general and move to specific--- (abstract just takes time to get acquainted) \pause

\invisible<1>{\begin{defn} A \alert{relation} is a set of ordered pairs.  A \alert{function} \alert{F} is a relation such that,
\begin{eqnarray}
(x, y) \in F & ; & (x, z) \in F \Rightarrow y = z \nonumber
\end{eqnarray}

We will commonly write a function as $F(x) $, where $x \in $ Domain $F$ and $F(x) \in $ Codomain $F$.  It is common to see people write,
\begin{equation}
F:A \rightarrow B \nonumber
\end{equation}
where $A$ is domain and $B$ is codomain
\end{defn}} \pause

\invisible<1-2>{Examples} \pause
\begin{itemize}
\invisible<1-3>{\item[-] $F(x) = x$ } \pause
\invisible<1-4>{\item[-] $F(x) = x^2$} \pause
\invisible<1-5>{\item[-] $F(x) = \sqrt{x}$}
\end{itemize}


\end{frame}

\begin{frame}
\frametitle{{\tt R} Computing Language}
\begin{itemize}
\item[-] We're going to use {\tt R} throughout the course
\item[-] {\tt R} as calculator :
\begin{itemize}
\item[] {\tt > 1 + 1}
\item[] {\tt [1] 2 }
\item[] {\tt > `Hello World' }
\item[] {\tt [1] ``Hello World"}
\end{itemize}
\item[-] {\tt object<- 2 \#\# assign numbers to objects}
\item[-] R has functions defined, we can define them to objects as well
\begin{itemize}
\item[] {\tt first.func<- function(x) \{ }
\item[] {\tt 	out<- 2*x }
\item[] {\tt return(out) \} }
\end{itemize}
\item[] {\tt first.func(2)  }
\item[] {\tt [1] 4 }
\end{itemize}







\end{frame}


\begin{frame}
\frametitle{Plotting Functions}

\begin{columns}[]

\column{0.7\textwidth}
\only<1>{\scalebox{0.6}{\includegraphics{f1.pdf}}}
\only<2>{\scalebox{0.6}{\includegraphics{f2.pdf}}}
\only<3>{\scalebox{0.6}{\includegraphics{f3.pdf}}}
\only<4>{\scalebox{0.6}{\includegraphics{f4.pdf}}}

\column{0.3\textwidth}
{\tt x<- seq(-2, 2, len=1000)}\\
\only<1>{{\tt plot(x$\sim$x) \#\# Results may vary}\\}

\only<2>{{\tt x.2<- x$*$x } \\}
\only<2>{{\tt plot(x.2$\sim$x) }\\}

 \only<3>{{\tt sin.2x<- sin(2$*$x)} \\}
\only<3>{{\tt plot(sin.2x$\sim$x) }\\}

 \only<4>{{\tt tanhx<- tanh(x)} \\}
\only<4>{{\tt plot(tanhx$\sim $x) }\\}

\end{columns}
\end{frame}


\begin{frame}
\frametitle{Exponents, Logarithms, and All That}

\pause
\begin{eqnarray}
\invisible<1>{f(x) & = &  2^{x} \nonumber \\ } \pause
\invisible<1-2>{g(x)  & = & e^{x}     \nonumber } \pause
\end{eqnarray}

\invisible<1-3>{Some rules of exponents remember $a$ could equal $e$} \pause
\begin{eqnarray}
\invisible<1-4>{a^{x}\times a^{y} & = & a^{x + y} \nonumber \\} \pause
\invisible<1-5>{\left(a^{x}\right)^{y} & = & a^{x\times y} \nonumber \\} \pause
\invisible<1-6>{\frac{a^{x}}{a^{y}} & = & a^{x - y} \nonumber \\} \pause
\invisible<1-7>{\frac{1}{a^{x} } & = & a^{-x} \nonumber \\} \pause
\invisible<1-8>{a^{x}\times b^{x} & = & (a \times b)^{x} \nonumber \\} \pause
\invisible<1-9>{a^{0} & = & 1 \nonumber \\} \pause
\invisible<1-10>{a^{1} & = & a \nonumber \\} \pause
\invisible<1-11>{1^{x} & = & 1 \nonumber }
\end{eqnarray}




\end{frame}


\begin{frame}
\frametitle{Exponents, Logarithms, and All That}

Logaritm $\log$ is a \alert{class} of functions.  \pause
\begin{itemize}
\invisible<1>{\item[-] $\log_{e} z  = $ what number $x$ solves $e^{x} = z$. } \pause
\invisible<1-2>{\item[-] We'll call $\log_{e}$ \alert{natural logarithm}.  And we'll assume $\log_{e} = \log$ } \pause
\invisible<1-3>{\item[-] $\log e  = 1$ (because $e^{1} = e$) } \pause
\invisible<1-4>{\item[-] $\log_{10} 1000 = 3$ (because $10^{3} = 1000$) } \pause
\end{itemize}

\invisible<1-5>{Some rules of logarithms } \pause
\begin{itemize}
\invisible<1-6>{\item[-] $\log(a \times b)  = \log(a) + \log (b) $ (\alert{!!!!!!}) } \pause
\invisible<1-7>{\item[-] $\log(\frac{a}{b}) = \log(a) - \log(b) $  } \pause
\invisible<1-8>{\item[-] $\log(a^{b} )  = b \log (a) $ } \pause
\invisible<1-9>{\item[-] $\log(1) = 0$ } \pause
\invisible<1-10>{\item[-] $\log(e) = 1$ }
\end{itemize}


\end{frame}

\begin{frame}
\frametitle{Properties of Functions}

Two important properties of functions \pause
\invisible<1>{\begin{defn}
A function $f:A \rightarrow B$ is 1-1 (one-to-one, or injective) if for all $y \in A$ and $z \in A$ in Domain, $f(y) = f(z)$ implies $y=z$.  In other words, preserves distinctiveness.
\end{defn}} \pause

\begin{itemize}
\invisible<1-2>{\item[-] $f(x) = x$} \pause
\invisible<1-3>{\item[-] $f(x) = x^2$} \pause
\end{itemize}

\invisible<1-4>{\begin{defn}
A function $f:A \rightarrow B$ is onto (surjective) if for all $b \in B$  there exists ($\exists$) $a \in A$ such that $f(a) = b$.
\end{defn}} \pause

\begin{itemize}
\invisible<1-5>{\item[-] $f:\{\hdots, -2, -1, 0, 1, 2, \hdots \} \rightarrow \{ 0, 1, 2, \hdots \}$ and $f(x) = |x|$.  onto, but not 1-1.} \pause
\invisible<1-6>{\item[-] $f:R \rightarrow R$ $f(x) = x$.  Onto and 1-1, \alert{bijective}}
\end{itemize}
\end{frame}

\begin{frame}
\frametitle{Composite Functions}

\begin{defn}
Suppose $f:A \rightarrow B$ and $g:B \rightarrow C$.  Then, define,
\begin{eqnarray}
g\circ f & = & g(f(x))\nonumber
\end{eqnarray}
\end{defn}

\begin{itemize}
\item[-] $f(x) = x$, $g(x) = x^2$.  Then $g \circ  f= x^2$.
\item[-] $f(x)  = \sqrt{x} $, $g(x)  =e^{x}$.  Then $g \circ f = e^{\sqrt{x}}$.
\item[-] $f(x) = sin(x)$, $g(x) = |x|$.  Then $g \circ f = |sin(x)|$.
\end{itemize}

\end{frame}



\begin{frame}
\frametitle{Inverse Function}

\begin{defn}
Suppose a function $f$ is 1-1.  Then we'll define $f^{-1}$ as its \alert{inverse} if,
\begin{eqnarray}
f^{-1}(f(x)) & = & x \nonumber
\end{eqnarray}

\end{defn}

\alert{Why do we need 1-1?}


\end{frame}

\begin{frame}
\frametitle{Induction}

\alert{Well Ordering Principle} Every non-empty set $J$ has a smallest number \pause

\invisible<1>{\begin{thm} If $P(n)$ is a statement containing the variable $n$ such that
\begin{itemize}
\item[i.] $P(1)$ is a true statement, and
\item[ii.] for each $k \in {1, 2, 3, 4, \hdots, n, \hdots}$ if $P(k)$ is true then $P(k+1)$ is true
\end{itemize}
then $P(n)$ is true for all $n  \in {1, 2, 3, 4, \hdots, n, \hdots}$
\end{thm}}


\end{frame}


\begin{frame}
\frametitle{Induction and Contradiction}


We'll use \alert{contradiction} and well ordering to prove that induction works.  \pause

\invisible<1>{\begin{proof}  Suppose $P(n)$ is some statement about the variable $n$ and that } \pause
\begin{itemize}
\invisible<1-2>{\item[i.] $P(1)$ is true } \pause
\invisible<1-3>{\item[ii.] If $P(k)$ is true then $P(k +1)$ is true.  } \pause
\end{itemize}

\invisible<1-4>{Now suppose, \alert{by way of contradiction} that there exists $N$ such that $P(N)$ is false.  This implies that } \pause
\begin{eqnarray}
\invisible<1-5>{S & = & \{x: P(x) \text{ is not true }\} \nonumber} \pause
\end{eqnarray}

\invisible<1-6>{By well ordering principle, there is smallest member of $S$, call it $n_{0}$.} \pause \invisible<1-7>{ By $i.$ we know that $n_{0}>1$.  Further, because $n_{0}$ is smallest member of $S$, then $P(n_{0})$ is false, but $P(n_{0}-1)$ is true.} \pause \invisible<1-8>{But now we have a problem,  because if $P(n_{0} - 1)$   is true, then $P(n_{0})$ is also true.} \pause \invisible<1-9>{ This implies that there is no smallest element of $S$.  \alert{CONTRADICTION}}
\end{proof}


\end{frame}

\begin{frame}
\frametitle{Summing $N$ numbers}
Induction is a useful proof technique.

\begin{thm} $\sum_{i=1}^{N} i  = 1 + 2 + 3 + 4 + \hdots + N = \frac{N (N +1) }{2} $
\end{thm}

\small

Two conditions to show:
\begin{itemize}
\item[i.] $\sum_{i=1}^{1}  i = 1 $ and $ \frac{1 (1 + 1) }{2}  = 1$
\end{itemize}




\end{frame}


\begin{frame}
\frametitle{Summing $N$ numbers}
\begin{itemize}
\item[ii.] Suppose true $N$.  Then, for $N+1$ we have,
\end{itemize}
\begin{eqnarray}
\sum_{i=1}^{N+1} i & = & \sum_{i=1}^{N} i  + (N + 1) \nonumber \\
							& = & \frac{N(N + 1)}{2}  + \frac{ 2 (N + 1)}{2} \nonumber \\
							& = & \frac{(N + 1)( N + 2) }{2} \nonumber \\
							& = & \frac{ (N + 1)( (N + 1) + 1)}{2} \nonumber
\end{eqnarray}

Conditions of induction met.  Therefore, proof complete

\end{frame}


\begin{frame}


{\tt Very Simple R Code}


\end{frame}


\begin{frame}
\frametitle{Finite, Countable, and Uncountable}
Three sizes of sets
\begin{itemize}
\item[1)] A set, $X$ is finite if there is a bijective function from $\{1, 2, 3, \hdots, n\}$ to $X$.
\item[2)] A set $X$ is \alert{countably infinite} if there is a bijective function from $\{1, 2, 3, 4, \hdots, \}$ to $X$.
\item[3)] A set $X$ is \alert{uncountably infinite} if it is not countable
\end{itemize}

The \alert{Real numbers} are \alert{uncountably infinite}


\end{frame}

\begin{frame}
\frametitle{Recap}
\alert{We've covered a lot}.\pause \\
\invisible<1>{\alert{PLEASE} don't worry---we're here to help!} \pause
\begin{itemize}
\invisible<1-2>{\item[1)] Sets + Operations} \pause
\invisible<1-3>{\item[2)] Functions} \pause
\invisible<1-4>{\item[3)] Contradiction, Induction, and direct proofs}
\end{itemize}







\end{frame}




\begin{frame}

Tomorrow:
\begin{itemize}
\item[-] Convergence of sequences
\item[-] Limits
\item[-] Continuity
\item[-] Derivatives
\end{itemize}
\end{frame}






























\end{document}
